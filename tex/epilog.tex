\chapter{Conclusion}

This thesis has proposed and implemented a genetic programming library in Swift. The library consists of multiple components, which can be used separately or in conjunction to facilitate easy implementation of efficient genetic algorithms operating on various data structures. The presented library is not dependent on any platform-specific components, and is thus fully portable to any operating system capable of supporting\footnote{At the time of writing this work, Swift runtime environment was supported on Apple Mac OS X, iOS and certain Linux distributions (Ubuntu, Debian, Gentoo, CentOS).} the Swift runtime environment. 

To demonstrate the practical usage of the library and its effortless integration with already existing third-party components, five example projects have been prepared and documented in its distribution package. Some of these projects were described in detail in the Chapter \ref{chapter:usage-demo} of this thesis.

In the last example, a strategy capable of playing the online game QWOP \cite{QwopWebsite} has been devised. By partially replicating the configuration described in \cite{EvolvingQwopGaits}, an algorithm built with the help of the library yielded results similar\footnote{The best strategy generated in this thesis was able to complete the QWOP race in approximately 2.5 minutes. One of the best-evolved gaits presented in \cite{EvolvingQwopGaits}, which used the same encoding, achieved the same distance in ``about 2 minutes.''} to those presented in the referenced publication.

The library source package including its documentation, unit tests and the mentioned examples is available online\footnote{The library website is: \url{https://github.com/petrmanek/Revolver}} or as a digital attachment to this thesis for free public use regulated by the standard MIT license agreement. Thanks to Swift's compatibility with iOS, the presented library has various applications in portable and wearable devices, for instance in the field of signal processing and gesture recognition.

In addition, the minimalistic syntax of Swift source codes allows students to easily examine the implementation within the source package of the library, learning about genetic algorithms in general. The library also supports its use in conjunction with Swift \textit{playground environments}, which enable quick prototyping of source codes with instantaneous visualization feedback from the compiler, making it useful in teaching.

