\chapter{Introduction}

\section{Evolutionary Algorithms}
\todo

\section{Genetic Programming}
\todo

\section{The Swift Programming Language}
In 2014, the Apple Corporation in California unveiled a new programming language called \textit{Swift}. \cite{SwiftReference} This language has been since then widely adopted by software developers and computer engineers, succeeding Objective-C as the main programming language used for application development on the Apple platform. Building on proven coding paradigms, such as generics and strongly-typed objects, Swift strives to be a modern, concise and safe alternative to popular languages like Python or C++ while attempting to maintain comparable performance in terms of computational speed and memory management (this has been observed experimentally \cite{PrimateLabsBenchmark}).

The latest version of the language is called Swift 2.2. Announced at the World Wide Developer Conference in 2015, the new Swift extended minimalistic syntax of its predecessor to include error handling, condition assertion and instruction deferring. It also enables developers to create frameworks, redistributable packages containing libraries that other developers can include and utilize in their projects.

Since Fall 2015, Swift along with its standard library became an open-source project, integrating itself into the LLVM project (Low-level Virtual Machine), a widely-used compiler compatible with thousands of devices running Linux-based operating systems. Thanks to these modifications, Swift has been lately discussed as an attractive option for academic collaborations and community-driven projects.

\section{Practical Application}
\todo

\section{Structure of This Document}
\todo
