\chapter{Object-oriented Design}
In this chapter, the high-level design of individual components of the library is described.

\cite{Koza1992}
\todo % odebrat toto, jakmile někde bude citace


\section{Random Genereration}
Randomness plays a crucial role in evolutionary algorithms. Since the properties of pseudo-random generators impact the quality of produced solutions significantly, the library gives users full control over the algorithm, which is used to produce random sequences. In object design, this is achieved by simple abstraction.

The functionality of a random number generator is facilitated by \textit{an entropy generator} object. In runtime, only a single instance of such object is created. This instance is then passed on to other components of the library, which require its capabilities. These components access the entropy generator by reference. Users are responsible for instantiating this object, and can thus specify a seed for the generator or choose an algorithm particularly suitable for their application.

For the sake of minimality, entropy generators are only required to produce positive floating-point decimals from the $[0;1]$ interval. In spite of that, they can be used to generate random values of various types. This mechanism provided that the generated decimals can be mapped onto the type while maintaining uniform distribution of generated values. This is further discussed in section \ref{section:data-structures}.


\subsection{Data Structures}\label{section:data-structures}
Every individual in a generation is repesented by a separate instance of a class. The primary responsibility of such object is to store genetic information, which defines the individual. This information does not need to be held in a homogeneous data structure. In fact, it can be stored in any type suitable for the application. The only requirement on such type is that it can be generated randomly.



\todo

\subsection{Randomizable Interface}
\todo

\subsection{Discrete Interface}
\todo

\section{Genetic Operators}
\todo

\subsection{Operator Life Cycle}
\todo

\subsection{Custom Interfaces}
\todo

\section{Selections}
\todo

\section{Algorithms}
\todo
