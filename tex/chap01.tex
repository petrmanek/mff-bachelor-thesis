\chapter{Background}
This chapter is dedicated to establishing and describing terms needed to understand the rest of the work.

\section{Genetic Algorithms}
Genetic algorithms (GA) are iterative randomized optimization techniques, which are inspired by the process of natural selection. In the context of GA, points in the domain space are likened to \textit{individuals} of a biological species. Every individual carries a \textit{chromosome}, which describes its location in the domain space, thus defining its properties and capabilities.

To initialize the GA, a \textit{population} of individuals with random chromosomes is generated. During single iteration of the GA, individuals in the population compete for their right to reproduce, favoring those who maximize the value of a \textit{fitness function} which customarily maps every individual to a value from the $[0;1]$ interval. At the end of the iteration, fit individuals are selected and allowed to produce an \textit{offspring population}, on which the next iteration of the GA operates.

\textit{The selection pressure} is the degree to which the better individuals are favored: the higher the selection pressure, the more the better individuals are favored. This selection pressure drives the GA to improve the population fitness over succeeding generations. However, if the selection pressure is too low, the convergence rate will be slow, and the GA will unnecessarily take longer to find the optimal solution. If the selection pressure is too high, there is an increased chance of the genetic algorithm prematurely converging to an incorrect (suboptimal) solution. \cite{GaTournamentSelection}

\section{Neural Networks}\label{section:neural-networks}
\todo


\begin{tikzpicture}[
plain/.style={
  draw=none,
  fill=none,
  },
net/.style={
  matrix of nodes,
  nodes={
    draw,
    circle,
    inner sep=10pt
    },
  nodes in empty cells,
  column sep=2cm,
  row sep=-9pt
  },
>=latex
]
\matrix[net] (mat)
{
|[plain]| \parbox{1.3cm}{\centering Input\\layer} & |[plain]| \parbox{1.3cm}{\centering Hidden\\layer} & |[plain]| \parbox{1.3cm}{\centering Output\\layer} \\
& |[plain]| \\
|[plain]| & \\
& |[plain]| \\
|[plain]| & |[plain]| \\
& & \\
|[plain]| & |[plain]| \\
& |[plain]| \\
|[plain]| & \\
& |[plain]| \\
};
\foreach \ai [count=\mi ]in {2,4,...,10}
  \draw[<-] (mat-\ai-1) -- node[above] {Input \mi} +(-2cm,0);
\foreach \ai in {2,4,...,10}
{\foreach \aii in {3,6,9}
  \draw[->] (mat-\ai-1) -- (mat-\aii-2);
}
\foreach \ai in {3,6,9}
  \draw[->] (mat-\ai-2) -- (mat-6-3);
\draw[->] (mat-6-3) -- node[above] {Ouput} +(2cm,0);
\end{tikzpicture}

\begin{tikzpicture}[
init/.style={
  draw,
  circle,
  inner sep=2pt,
  font=\Huge,
  join = by -latex
},
squa/.style={
  draw,
  inner sep=2pt,
  font=\Large,
  join = by -latex
},
start chain=2,node distance=13mm
]
\node[on chain=2] 
  (x2) {$x_2$};
\node[on chain=2,join=by o-latex] 
  {$w_2$};
\node[on chain=2,init] (sigma) 
  {$\displaystyle\Sigma$};
\node[on chain=2,squa,label=above:{\parbox{2cm}{\centering Activate \\ function}}]   
  {$f$};
\node[on chain=2,label=above:Output,join=by -latex] 
  {$y$};
\begin{scope}[start chain=1]
\node[on chain=1] at (0,1.5cm) 
  (x1) {$x_1$};
\node[on chain=1,join=by o-latex] 
  (w1) {$w_1$};
\end{scope}
\begin{scope}[start chain=3]
\node[on chain=3] at (0,-1.5cm) 
  (x3) {$x_3$};
\node[on chain=3,label=below:Weights,join=by o-latex] 
  (w3) {$w_3$};
\end{scope}
\node[label=above:\parbox{2cm}{\centering Bias \\ $b$}] at (sigma|-w1) (b) {};

\draw[-latex] (w1) -- (sigma);
\draw[-latex] (w3) -- (sigma);
\draw[o-latex] (b) -- (sigma);

\draw[decorate,decoration={brace,mirror}] (x1.north west) -- node[left=10pt] {Inputs} (x3.south west);
\end{tikzpicture}


\section{Redistributable Applications}
\todo

\section{Analysis}
\todo

\section{Requirements}
\todo

% \chapter{Object-oriented Design}
% In this chapter, the high-level design of individual components of the library is described.

% \cite{Koza1992}
% \todo % odebrat toto, jakmile někde bude citace


% \section{Random Genereration}
% Randomness plays a crucial role in evolutionary algorithms. Since the properties of pseudo-random generators impact the quality of produced solutions significantly, the library gives users full control over the algorithm, which is used to produce random sequences. In object design, this is achieved by simple abstraction.

% The functionality of a random number generator is facilitated by \textit{an entropy generator} object. In runtime, only a single instance of such object is created. This instance is then passed on to other components of the library, which require its capabilities. These components access the entropy generator by reference. Users are responsible for instantiating this object, and can thus specify a seed for the generator or choose an algorithm particularly suitable for their application.

% For the sake of minimality, entropy generators are only required to produce positive floating-point decimals from the $[0;1]$ interval. In spite of that, they can be used to generate random values of various types. This mechanism provided that the generated decimals can be mapped onto the type while maintaining uniform distribution of generated values. This is further discussed in section \ref{section:data-structures}.


% \subsection{Data Structures}\label{section:data-structures}
% Every individual in a generation is repesented by a separate instance of a class. The primary responsibility of such object is to store genetic information, which defines the individual. This information does not need to be held in a homogeneous data structure. In fact, it can be stored in any type suitable for the application. The only requirement on such type is that it can be generated randomly.



% \todo

% \subsection{Randomizable Interface}
% \todo

% \subsection{Discrete Interface}
% \todo

% \section{Genetic Operators}
% \todo

% \subsection{Operator Life Cycle}
% \todo

% \subsection{Custom Interfaces}
% \todo

% \section{Selections}
% \todo

% \section{Algorithms}
% \todo
