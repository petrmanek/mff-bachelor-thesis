\chapter{Introduction}

Genetic algorithms represent a class of non-deterministic machine learning techniques inspired by the processes occurring in the nature (hence the name). Originally proposed in 1950s, these techniques have proven to be efficient solutions to some optimization and search problems, especially in cases where the domain space is poorly understood, unpredictable or irregular.

In the mentioned cases, genetic algorithms have become attractive heu\-ristics for tasks which would otherwise require purely exponential solutions. Thanks to ongoing advances in computer performance, genetic algorithms have found various applications, facilitating their use in automatic code generation for real-time control systems and classifiers, signal processing and NP-hard combinatorial problem solving.

In 2014, the Apple Corporation unveiled a new programming language called \textit{Swift} \cite{SwiftReference}. This language has been since then widely adopted by software developers and computer engineers, succeeding Objective-C as the main programming language used for application development on the Apple platform. Building on proven coding paradigms, such as generics and strongly-typed objects, Swift strives to be a modern, concise and safe alternative to popular languages like Python or C++ while attempting to maintain comparable performance in terms of computational speed and memory management (this has been observed experimentally \cite{PrimateLabsBenchmark}).

The latest version of the language is called \textit{Swift 2.2}. Announced at the World Wide Developer Conference in 2015, the new Swift extended minimalistic syntax of its predecessor to include error handling, condition assertion and instruction deferring. It also enables developers to create \textit{frameworks}, redistributable packages containing documentation and binary libraries that other developers can include and utilize in their projects.

Since Fall 2015, Swift along with its standard library\footnote{The Swift standard library is comprised of two main components, the \textit{Foundation} module and the \textit{libdispatch} scheduling library.} became an open-source project, integrating itself into the LLVM (Low-level Virtual Machine), a widely-used compiler compatible with thousands of devices running Linux-based operating systems. Thanks to these modifications, Swift has been lately discussed as an attractive option for academic collaborations and community-driven projects.

The subject of this thesis is the implementation of an open-source library, which would add the support for building genetic algorithms to Swift. Although there have already been isolated attempts at implementing variants of genetic algorithms in this programming language, such works have served primarily for the purposes of demonstration. In comparison, the aim of this work is to offer an efficient and highly extensible basis for developing applications that harness the power of genetic algorithms and put it to use in common scenarios.

To better demonstrate the utilization of the presented library, this thesis includes several instances of examples, in which the components of the library are applied to solve practical problems. The most notable of these examples is the development of an artificial player capable of achieving human-competitive scores in a popular online game.

This thesis is organized as follows.

Chapter \ref{chapter:background} serves to define ground terms used in this document and link them to their counterparts in the cited literature.

Chapter \ref{chapter:feature-analysis} is a brief summary of the features offered by other similar works concluded by the statement of requirements on the library presented by this work.

Chapter \ref{chapter:documentation} provides details on the object design of the presented library. It introduces its individual components, explains their purpose and gives recommendations on their usage. In addition, the chapter includes short Swift code fragments to further illustrate some referenced programming techniques. 

Chapter \ref{chapter:usage-demo} is dedicated to practical usage demonstrations of the library as a whole. It contains three demonstrations selected from a broader list of example projects which are part of the library source package. 

Chapter \ref{chapter:conclusion} contains the conclusion of this thesis. It summarizes its objectives, states its accomplishments and concludes by suggesting potential applications of the library in research, portable devices and teaching.

Appendix \ref{appendix:survey} is a survey of other GA libraries, conducted in September 2015 as a part of preliminary specification, which eventually led to the writing of this work.

The digital version of this thesis also contains Appendix B, which contains the library source package with implementation, unit tests, documentation and five example projects.
