\chapter{Conclusion}~\label{chapter:conclusion}

This thesis has managed to satisfy all of the goals set forth in Section~\ref{section:requirements}. It has proposed and implemented a genetic programming library in Swift and successfully applied it to three sample problems. One of the mentioned problems was a human-competitive task, which was solved with results comparable to \cite{EvolvingQwopGaits}.

The library itself consists of multiple components, which can be used separately or in conjunction to facilitate easy implementation of efficient genetic algorithms operating on various data structures. The presented library is not dependent on any platform-specific components, and is thus fully portable to any operating system capable of supporting\footnote{At the time of writing this work, Swift runtime environment was supported on Apple Mac OS X, iOS and certain Linux distributions (Ubuntu, Debian, Gentoo, CentOS).} the Swift standard runtime environment. 

To demonstrate the practical usage of the library and its effortless integration with already existing third-party components, five example projects have been prepared (including those described in Chapter \ref{chapter:usage-demo}) and bundled in its distribution package with the library's documentation and unit tests. The mentioned package is available online\footnote{The library website is: \url{https://github.com/petrmanek/Revolver}} or as a digital attachment to this thesis for free public use regulated by the standard MIT license agreement.

Thanks to Swift's compatibility with iOS, the library has various applications in portable and wearable devices, for instance in the field of signal processing and gesture recognition. In addition, the minimalistic syntax of Swift source codes allows students to easily examine the implementation within the source package, learning about genetic algorithms in general. The library also supports usage within Swift \textit{playground environments}, which enable quick prototyping of source codes with instantaneous visualization feedback from the compiler, making it useful in teaching.

Lastly, the presented library has been designed to have modular architecture. It is the author's intention that its components are open to further extensions by its users.

