\chapter{Feature Analysis}~\label{chapter:feature-analysis}

This chapter includes a brief overview of the features offered by other works (Section~\ref{section:survey-of-previous-works}) and concludes with a statement of requirements on the presented library (Section~\ref{section:requirements}).

\section{Survey of Previous Works}~\label{section:survey-of-previous-works}
In September 2015, prior to writing this thesis, an online survey of over 30 redistributable GA libraries was conducted. Its results are attached in Appendix \ref{appendix:survey}.

In general, the survey seemed to indicate that there exist many implementations of GA in historically old programming languages and only a few that are compatible with the new ones, such as Swift.

Due to the high number of projects, libraries written in well-established languages offered more specialized features, focusing for instance only on grammar evolution or distributed evaluation. On the other hand, implementations in young or still-evolving languages focused more on providing a stable runtime foundation for GA execution.

\subsection{Swift Implementations}
In Swift, only two working implementations were discovered. The first implementation is \textit{Genetic Swift}\footnote{Genetic Swift is available online: \url{https://github.com/NikoYuwono/Genetic-Swift}}, a single-file demonstration application of the fundamental principles of GA without any focus on customization or redistributablity in other projects.

The other implementation was \textit{Mendel}\footnote{Mendel is available online: \url{https://github.com/saniul/Mendel}}, a small yet highly extensible framework for implementing GA. Although Mendel met the semantic criteria of the survey, one might hesitate to describe it as a robust foundation for developing and executing GA mainly due to its minimalistic design which complies more with the guidelines of functional than object-oriented programming.

\section{Statement of Requirements}~\label{section:requirements}
With the results of the survey in mind, this section states key requirements on the GA library, which is the subject of this thesis.
~
\begin{enumerate}
	\item
	Divide the GA into separate components in compliance with the defined terms and the referenced literature.

	\item
	Design an object-oriented model and state requirements and assumptions for every of its components.

	\item 
	Provide implementation of at least one component of every type, possibly also implementing other frequently used alternatives.

	\item 
	Allow customization of components by the means of conventional object polymorphism, Swift generics and extensions.

	\item
	Offer documentation and unit tests of the provided implementation (where applicable).

	\item
	Create at least three working usage examples of practical application.
\end{enumerate}
